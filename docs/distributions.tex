\subsection*{Autoregressive Normal} 
\verb=arnormal_like(x, a, sigma, rho)=
\begin{eqnarray*}
	x_i & = a_i \exp(e_i) \\
   e_i & = \rho e_{i-1} + \epsilon_i \\
 \\
 & \epsilon_i \sim N(0,\tau)	
\end{eqnarray*}

\subsection*{Bernoulli}
\verb=bernoulli_like(x, p)=
\begin{eqnarray*}
f(x \mid p) &= p^{x- 1} (1-p)^{1-x} \\
\\
&0 < p < 1 \\
&x=0,1
\end{eqnarray*}

\subsection*{Beta}
\verb=beta_like(x, =$\alpha$\verb=, =$\beta$\verb=)=
\begin{eqnarray*}
f(x \mid \alpha, \beta) &=& \frac{\Gamma(\alpha + \beta)}{\Gamma(\alpha) \Gamma(\beta)} x^{\alpha - 1} (1 - x)^{\beta - 1} \\
\\
&&0 < x < 1 \\
&&\alpha > 0, \beta > 0
\end{eqnarray*}

\subsection*{Binomial}
\verb=binomial_like(x, n, p)=
\begin{eqnarray*}
f(x \mid n, p) &=& \frac{n!}{x!(n-x)!}p^x (1-p)^{1-x} \\
\\
&&0 < p < 1 \\
&&x = 0,\ldots,n
\end{eqnarray*}

\subsection*{Cauchy}
\verb=cauchy_like(x, =$\alpha$\verb=, =$\beta$\verb=)=
\begin{eqnarray*}
f(x \mid \alpha, \beta) &=& \frac{1}{\pi \beta [1 + (\frac{x-\alpha}{\beta})^2]} \\
\\
&& \beta > 0
\end{eqnarray*}

\subsection*{Chi-squared}
\verb=chi2_like(x, =$\nu$\verb=)=
\begin{eqnarray*}
f(x \mid \nu) &=& \frac{x^{(\nu-2)/2}e^{-x/2}}{2^{\nu/2}\Gamma(\nu/2)} \\
\\
&& x \ge 0 \\
&& \nu > 0
\end{eqnarray*}

\subsection*{Dirichlet}
\verb=dirichlet_like(p, =$\theta$\verb=)=
\begin{eqnarray*}
f(\mathbf{p}) &=& \frac{\Gamma(\sum_{i=1}^k \theta_i)}{\prod_{i=1}^k \Gamma(\theta_i)} \prod_{i=1}^k p_i^{\theta_i - 1}\\
\\
&& 0 < p_i < 1 \\
&& \theta_i > 0
\end{eqnarray*}

\subsection*{Exponential}
\verb=exponential_like(x, =$\alpha$\verb=, =$\beta$\verb=)=
\begin{eqnarray*}
f(x \mid \beta) &=& \frac{1}{\beta}e^{-x/\beta} \\
\\
&& x \ge 0 \\
&& \beta > 0
\end{eqnarray*}

\subsection*{Gamma}
\verb=gamma_like(x, =$\alpha$\verb=, =$\beta$\verb=)=
\begin{eqnarray*}
f(x \mid \alpha, \beta) &=& \frac{x^{\alpha-1}e^{-x/\beta}}{\Gamma(\alpha) \beta^{\alpha}} \\
\\
&& x \ge 0 \\
&& \alpha > 0 \\
&& \beta > 0
\end{eqnarray*}

\subsection*{Generalized Extreme Value}
\verb=gev_like(x, =$\alpha$\verb=, =$\beta$\verb=)=
\begin{eqnarray*}
f(x \mid \xi,\mu,\sigma) &=& \frac{1}{\sigma}(1 + \xi \left[\frac{x-\mu}{\sigma}\right])^{-1/\xi-1}\exp{-(1+\xi \left[\frac{x-\mu}{\sigma}\right])^{-1/\xi}}\\
\\
&& \sigma > 0,\\
&& x > \mu-\sigma/\xi \text{ if } \xi > 0,\\
&& x < \mu-\sigma/\xi \text{ if } \xi < 0\\
&& x \in [-\infty,\infty] \text{ if } \xi = 0
\end{eqnarray*}

\subsection*{Geometric}
\verb=geometric_like(x, p)=
\begin{eqnarray*}
f(x \mid p) &=& p(1-p)^{x-1} \\
&& 0 < p < 1 \\
&& x = 1,2,3,\ldots
\end{eqnarray*}

\subsection*{Half-normal}
\verb=half_normal_like(x, =$\tau$\verb=)=
\begin{eqnarray*}
f(x \mid \tau) &=& \sqrt{\frac{2\tau}{\pi}} \exp\left\{ {\frac{-x^2 \tau}{2}}\right\} \\
\vspace{3cm}\\
&& x \ge 0 \\
&& \tau > 0
\end{eqnarray*}

\subsection*{Hypergeometric}
\verb=hypergeometric_like(x,n,m,N)=
\begin{eqnarray*}
f(x \mid n, m, N) &=& \frac{\left(\begin{array}{c}m \\x\end{array}\right) \left(\begin{array}{c}N-m \\n-x\end{array}\right)}{\left(\begin{array}{c}N \\n\end{array}\right)} \\
\vspace{3cm}\\
&& x = 0,1,\ldots,n\\
&& m = 0,1,\ldots,N
\end{eqnarray*}

\subsection*{Inverse gamma}
\verb=inverse_gamma_like(x,=$\alpha$\verb=,=$\beta$\verb=)=
\begin{eqnarray*}
f(x \mid \alpha, \beta) &=& \frac{x^{-\alpha - 1} e^{-\frac{1}{x\beta}}}{\Gamma(\alpha)\beta^\alpha} \\
\\
&& x \ge 0 \\
&& \alpha > 0 \\
&& \beta > 0
\end{eqnarray*}

\subsection*{Log-normal}
\verb=log_normal_like(x, =$\mu$\verb=, =$\tau$\verb=)=
\begin{eqnarray*}
f(x \mid \mu, \tau) &=& \sqrt{\frac{\tau}{2\pi}} \exp\left\{ -\frac{\tau}{2} (x-\mu)^2 \right\}\\
\\
&& \tau > 0
\end{eqnarray*}

\subsection*{Multinomial}
\verb=multinomial_like(x, n, p)=
\begin{eqnarray*}
f(x \mid n, p) &=& \frac{n!}{\prod_{i=1}^k x_i!} \prod_{i=1}^k p_i^{x_i}\\
\\
&& \sum_{i=1}^k x_i=n \\
&& \sum_{i=1}^k p_i=1
\end{eqnarray*}

\subsection*{Multivariate normal}
\verb=multivariate_normal_like(x, =$\pi$\verb=, =$\tau$\verb=)=
\begin{eqnarray*}
f(x \mid \pi, T) &=& \frac{T^{n/2}}{(2\pi)^{1/2}} \exp\left\{ -\frac{1}{2} (x-\mu)^{\prime}T(x-\mu) \right\}\\
\\
&& \verb=T positive definite=
\end{eqnarray*}

\subsection*{Multivariate hypergeometric}
\verb=multivariate_hypergeometric_like(x, n, m, N)=
\begin{eqnarray*}
f(x \mid n, m, N) &=& \frac{\prod_{i=1}^k \left(\begin{array}{c}m_i \\x_i\end{array}\right)}{\left(\begin{array}{c}N \\n\end{array}\right)} \\
\vspace{3cm}\\
&& \sum_{i=1}^k x_i = n\\
&& \sum_{i=1}^k m_i = N
\end{eqnarray*}

\subsection*{Negative binomial}
\verb=negative_binomial_like(x, r, p)=
\begin{eqnarray*}
f(x \mid r, p) &=& \frac{(x+r-1)!}{x! (r-1)!} p^r (1-p)^x \\
\\
&&0 < p < 1 \\
&&x = 0,1,2,\ldots \\
&&r=1,2,3,\ldots
\end{eqnarray*}

\subsection*{Normal}
\verb=normal_like(x, =$\mu$\verb=, =$\tau$\verb=)=
\begin{eqnarray*}
f(x \mid \mu, \tau) &=& \sqrt{\frac{\tau}{2\pi}} \exp\left\{ -\frac{\tau}{2} (x-\mu)^2 \right\}\\
\\
&& \tau > 0
\end{eqnarray*}

\subsection*{Poisson}
\verb=poisson_like(x, =$\mu$\verb=)=
\begin{eqnarray*}
f(x \mid \mu) &=& \frac{e^{-\mu}\mu^x}{x!}\\
\\
&& \mu > 0\\
&& x = 0,1,2,\ldots
\end{eqnarray*}

\subsection*{Uniform}
\verb=uniform_like(x, a, b)=
\begin{eqnarray*}
f(x \mid a, b) &=& \frac{1}{b-a}\\
\\
&& a \le x \le b
\end{eqnarray*}

\subsection*{Uniform mixture}
\verb=uniform_mixture_like(x, a, m, b)=
\begin{eqnarray*}
f(x \mid a, m, b) &=& \left\{\begin{array}{c}\frac{0.5}{m-a}, \hspace{0.1cm} a \le x < m \\
\\
\frac{0.5}{b-m}, \hspace{0.1cm}m \le x \le b \end{array}\right.
\end{eqnarray*}

\subsection*{Weibull}
\verb=weibull_like(x, =$\alpha$\verb=, =$\beta$\verb=)=
\begin{eqnarray*}
f(x \mid \alpha, \beta) &=& \frac{\alpha x^{\alpha - 1} \exp(-(\frac{x}{\beta})^{\alpha})}{\beta^\alpha} \\
\\
&& x \ge 0 \\
&& \alpha > 0 \\
&& \beta > 0
\end{eqnarray*}

\subsection*{Wishart}
\verb=wishart_like(X, n, T)=
\begin{eqnarray*}
f(X \mid n, T) &=& {\mid T \mid}^{n/2}{\mid X \mid}^{(n-k-1)/2} \exp\left\{ -\frac{1}{2} Tr(TX) \right\}\\
\\
&& \verb=X,T symmetric and positive definite= \\
&& k = \verb=length=(X)
\end{eqnarray*}

\subsection*{Wrapped Cauchy}
\verb=wrapped_cauchy_like(x, =$\mu$\verb=, =$\rho$\verb=)=
\begin{eqnarray*}
f(x \mid \mu, \rho) &=& \frac{1-\rho^2}{2\pi [1 + \rho^2 - 2\rho\cos(x-\mu)]} \\
\\
&& 0 \le \mu < 2\pi \\
&& 0 \le \rho \le 1
\end{eqnarray*}
