PyMC tries to make standard things easy, but keep unusual things possible. Its openness, combined with Python's flexibility, invite extensions from using new step methods to exotic stochastic processes (see the Gaussian process module). This chapter briefly reviews the ways PyMC is designed to be extended.

\hypertarget{nonstandard}{}
\section{Nonstandard Stochastics} \label{nonstandard}
\pdfbookmark[0]{Nonstandard Stochastics}{nonstandard}

The simplest way to create a \code{Stochastic} object with a nonstandard distribution is to use the medium or long decorator syntax. See chapter \ref{chap:modelbuilding}. If you want to create many stochastics with the same nonstandard distribution, the decorator syntax can become cumbersome. An actual subclass of \code{Stochastic} can be created using the class factory \code{stochastic_from_dist}. This function takes the following arguments:
\begin{itemize}
   \item The name of the new class,
   \item A \code{logp} function,
   \item A \code{random} function,
   \item The NumPy datatype of the new class (for continuous distributions, this should be \code{float}; for discrete distributions, \code{int}; for variables valued as non-numerical objects, \code{object}),
   \item A flag indicating whether the resulting class represents a vector-valued variable.
\end{itemize}
The necessary parent labels are read from the \code{logp} function, and a docstring for the new class is automatically generated. Instances of the new class can be created in one line.

Full subclasses of \code{Stochastic} may be necessary to provide nonstandard behaviors (see \code{gp.GP}).

\hypertarget{custom-stepper}{}
\section{User-defined step methods} \label{custom-stepper}
\pdfbookmark[0]{User-defined step methods}{custom-stepper}
The \code{StepMethod} class is meant to be subclassed. There are an enormous number of MCMC step methods in the literature, whereas PyMC provides only about half a dozen. Most user-defined step methods will be either Metropolis-Hastings or Gibbs step methods, and these should subclass \code{Metropolis} or \code{Gibbs} respectively. More unusual step methods should subclass \code{StepMethod} directly.

\hypertarget{custom_stepper_example}{}
\subsection{Example: an asymmetric Metropolis step} \label{user-gen}
Consider the probability model in \file{examples/custom_step.py}:
\begin{verbatim}
mu = pymc.Normal('mu',0,.01, value=0)
tau = pymc.Exponential('tau',.01, value=1)
cutoff = pymc.Exponential('cutoff',1, value=1.3)
D = pymc.TruncatedNormal('D',mu,tau,-numpy.inf,cutoff,value=data,observed=True)
\end{verbatim}
The stochastic variable \code{cutoff} cannot be smaller than the largest element of $D$, otherwise $D$'s density would be zero. The standard \code{Metropolis} step method can handle this case without problems; it will propose illegal values occasionally, but these will be rejected.

\medskip
Suppose we want to handle \code{cutoff} with a smarter step method that doesn't propose illegal values. Specifically, we want to use the nonsymmetric proposal distribution
\begin{eqnarray*}
	x_p | x \sim \textup{Truncnorm}(x, \sigma, \max(D), \infty).
\end{eqnarray*}
We can implement this Metropolis-Hastings algorithm with the following step method class:
\begin{verbatim}
class TruncatedMetropolis(pymc.Metropolis):
    def __init__(self, stochastic, low_bound, up_bound, *args, **kwargs):
        self.low_bound = low_bound
        self.up_bound = up_bound
        pymc.Metropolis.__init__(self, stochastic, *args, **kwargs)

    # Propose method written by hacking Metropolis.propose()
    def propose(self):
        tau = 1./(self.adaptive_scale_factor * self.proposal_sd)**2
        self.stochastic.value = \
        pymc.rtruncnorm(self.stochastic.value, tau, self.low_bound, self.up_bound)

    # Hastings factor method accounts for asymmetric proposal distribution
    def hastings_factor(self):
        tau = 1./(self.adaptive_scale_factor * self.proposal_sd)**2
        cur_val = self.stochastic.value
        last_val = self.stochastic.last_value

        lp_for = pymc.truncnorm_like(cur_val, last_val, tau, self.low_bound, self.up_bound)
        lp_bak = pymc.truncnorm_like(last_val, cur_val, tau, self.low_bound, self.up_bound)

        if self.verbose > 1:
            print self._id + ': Hastings factor %f'%(lp_bak - lp_for)
        return lp_bak - lp_for
\end{verbatim}

The \code{propose} method sets the step method's stochastic's value to a new value, drawn from a truncated normal distribution. The precision of this distribution is computed from two factors: \code{self.proposal_sd}, which can be set with an input argument to Metropolis, and \code{self.adaptive_scale_factor}. Metropolis step methods' default tuning behavior is to reduce \code{adaptive_scale_factor} if the acceptance rate is too low, and to increase \code{adaptive_scale_factor} if it is too high. By incorporating \code{adaptive_scale_factor} into the proposal standard deviation, we avoid having to write our own tuning infrastructure. If we don't want the proposal to tune, we don't have to use \code{adaptive_scale_factor}.

The \code{hastings_factor} method adjusts for the asymmetric proposal distribution \cite{gelman}. It computes the log of the quotient of the `backward' density and the `forward' density. For symmetric proposal distributions, this quotient is 1, so its log is zero. We have added some code to print the Hastings factor if the step method's verbosity level is set high.

\medskip
Having created our custom step method, we need to tell MCMC instances to use it to handle the variable \code{cutoff}. This is done in \file{custom_step.py} with the following line:
\begin{verbatim}
M.use_step_method(TruncatedMetropolis, cutoff, D.value.max(), numpy.inf)
\end{verbatim}
This call causes $M$ to pass the arguments \code{cutoff, D.value.max(), numpy.inf} to a \code{TruncatedMetropolis} object's \code{init} method, and use the object to handle \code{cutoff}.

\medskip
It's often convenient to get a handle to a custom step method instance directly for debugging purposes. \code{M.step_method_dict[cutoff]} returns a list of all the step methods $M$ will use to handle \code{cutoff}:
\begin{verbatim}
>>> M.step_method_dict[cutoff]
[<custom_step.TruncatedMetropolis object at 0x3c91130>]
\end{verbatim}
There may be more than one, and conversely step methods may handle more than one stochastic variable. To see which variables step method $S$ is handling, try
\begin{verbatim}
>>> S.stochastics
set([<pymc.distributions.Exponential 'cutoff' at 0x3cd6b90>])
\end{verbatim}

\hypertarget{user-gen}{}
\subsection{General step methods} \label{user-gen}
\pdfbookmark[1]{General step methods}{user-gen}

All step methods must implement the following methods:
\begin{description}
   \item[\code{step()}:] Updates the values of \code{self.stochastics}.
   \item[\code{tune()}:] Tunes the jumping strategy based on performance so far. A default method is available that increases \code{self.adaptive_scale_factor} (see below) when acceptance rate is high, and decreases it when acceptance rate is low. This method should return \code{True} if additional tuning will be required later, and \code{False} otherwise.
   \item[\code{competence(s):}] A class method that examines stochastic variable $s$ and returns a value from 0 to 3 expressing the step method's ability to handle the variable. This method is used by \code{MCMC} instances when automatically assigning step methods. Conventions are:
   \begin{description}
      \item[0] I cannot safely handle this variable.
      \item[1] I can handle the variable about as well as the standard \code{Metropolis} step method.
      \item[2] I can do better than \code{Metropolis}.
      \item[3] I am the best step method you are likely to find for this variable in most cases.
   \end{description}
   For example, if you write a step method that can handle \code{MyStochasticSubclass} well, the competence method might look like this:
\begin{verbatim}
class MyStepMethod(pymc.StepMethod):
   def __init__(self, stochastic, *args, **kwargs):
      ...

   @classmethod
   def competence(self, stochastic):
      if isinstance(stochastic, MyStochasticSubclass):
         return 3
      else:
         return 0
\end{verbatim}
   Note that PyMC will not even attempt to assign a step method automatically if its \code{init} method cannot be called with a single stochastic instance, that is \code{MyStepMethod(x)} is a legal call. The list of step methods that PyMC will consider assigning automatically is called \code{pymc.StepMethodRegistry}.
   \item[\code{current_state()}:] This method is easiest to explain by showing the code:
   \begin{verbatim}
state = {}
for s in self._state:
    state[s] = getattr(self, s)
return state
   \end{verbatim}
   \code{self._state} should be a list containing the names of the attributes needed to reproduce the current jumping strategy. If an \code{MCMC} object writes its state out to a database, these attributes will be preserved. If an \code{MCMC} object restores its state from the database later, the corresponding step method will have these attributes set to their saved values.
\end{description}

Step methods should also maintain the following attributes:
\begin{description}
   \item[\code{_id}:] A string that can identify each step method uniquely (usually something like \code{<class_name>_<stochastic_name>}).
   \item[\code{adaptive_scale_factor}:] An `adaptive scale factor'. This attribute is only needed if the default \code{tune()} method is used.
   \item[\code{_tuning_info}:] A list of strings giving the names of any tuning parameters. For \texttt{Metropolis} instances, this would be \texttt{adaptive_scale_factor}. This list is used to keep traces of tuning parameters in order to verify `diminishing tuning' \cite{tuning}.
\end{description}

All step methods have a property called \code{loglike}, which returns the sum of the log-probabilities of the union of the extended children of \code{self.stochastics}. This quantity is one term in the log of the Metropolis-Hastings acceptance ratio.


\hypertarget{user-metro}{}
\subsection{Metropolis-Hastings step methods} \label{user-metro}
\pdfbookmark[1]{Metropolis-Hastings step methods}{user-metro}
A Metropolis-Hastings step method only needs to implement the following methods, which are called by \code{Metropolis.step()}:
\begin{description}
   \item[\code{reject()}:] Usually just
   \begin{verbatim}
def reject(self):
    self.rejected += 1
    [s.value = s.last_value for s in self.stochastics]
   \end{verbatim}
   \item[\code{propose():}] Sets the values of all \code{self.stochastics} to new, proposed values. This method may use the \code{adaptive_scale_factor} attribute to take advantage of the standard tuning scheme.
\end{description}
Metropolis-Hastings step methods may also override the \code{tune} and \code{competence} methods.

Metropolis-Hastings step methods with asymmetric jumping distributions may implement a method called \code{hastings_factor()}, which returns the log of the ratio of the `reverse' and `forward' proposal probabilities. Note that no \code{accept()} method is needed or used.

By convention, Metropolis-Hastings step methods use attributes called \code{accepted} and \code{rejected} to log their performance.

\hypertarget{user-gibbs}{}
\subsection{Gibbs step methods} \label{user-gibbs}
\pdfbookmark[1]{Gibbs step methods}{user-gibbs}

Gibbs step methods handle conjugate submodels. These models usually have two components: the `parent' and the `children'. For example, a gamma-distributed variable serving as the precision of several normally-distributed variables is a conjugate submodel; the gamma variable is the parent and the normal variables are the children.

This section describes PyMC's current scheme for Gibbs step methods, several of which are in a semi-working state in the sandbox. It is meant to be as generic as possible to minimize code duplication, but it is admittedly complicated. Feel free to subclass StepMethod directly when writing Gibbs step methods if you prefer.

Gibbs step methods that subclass PyMC's \code{Gibbs} should define the following class attributes:
\begin{description}
   \item[\code{child_class}:] The class of the children in the submodels the step method can handle.
   \item[\code{parent_class}:] The class of the parent.
   \item[\code{parent_label}:] The label the children would apply to the parent in a conjugate submodel. In the gamma-normal example, this would be \code{tau}.
   \item[\code{linear_OK}:] A flag indicating whether the children can use linear combinations involving the parent as their actual parent without destroying the conjugacy.
\end{description}

A subclass of \code{Gibbs} that defines these attributes only needs to implement a \code{propose()} method, which will be called by \code{Gibbs.step()}. The resulting step method will be able to handle both conjugate and `non-conjugate' cases. The conjugate case corresponds to an actual conjugate submodel. In the nonconjugate case all the children are of the required class, but the parent is not. In this case the parent's value is proposed from the likelihood and accepted based on its prior. The acceptance rate in the nonconjugate case will be less than one.

The inherited class method \code{Gibbs.competence} will determine the new step method's ability to handle a variable $x$ by checking whether:
\begin{itemize}
   \item all $x$'s children are of class \code{child_class}, and either apply \code{parent_label} to $x$ directly or (if \code{linear_OK=True}) to a \code{LinearCombination} object (chapter \ref{chap:modelbuilding}), one of whose parents contains $x$.
   \item $x$ is of class \code{parent_class}
\end{itemize}
If both conditions are met, \code{pymc.conjugate_Gibbs_competence} will be returned. If only the first is met, \code{pymc.nonconjugate_Gibbs_competence} will be returned.

\hypertarget{custom-model}{}
\section{New fitting algorithms} \label{custom-model}
\pdfbookmark[0]{New fitting algorithms}{custom-model}

PyMC provides a convenient platform for non-MCMC fitting algorithms in addition to MCMC. All fitting algorithms should be implemented by subclasses of \code{Model}. There are virtually no restrictions on fitting algorithms, but many of \code{Model}'s behaviors may be useful. See chapter \ref{chap:modelfitting}.

\hypertarget{custom-MC}{}
\subsection{Monte Carlo fitting algorithms} \label{custom-MC}
\pdfbookmark[1]{Monte Carlo fitting algoriths}{custom-MC}

Unless there is a good reason to do otherwise, Monte Carlo fitting algorithms should be implemented by subclasses of \code{Sampler} to take advantage of the interactive sampling feature and database backends. Subclasses using the standard \code{sample()} and \code{isample()} methods must define one of two methods:
\begin{description}
   \item[\code{draw()}:] If it is possible to generate an independent sample from the posterior at every iteration, the \code{draw} method should do so. The default \code{_loop} method can be used in this case.
   \item[\code{_loop()}:] If it is not possible to implement a \code{draw()} method, but you want to take advantage of the interactive sampling option, you should override \code{_loop()}. This method is responsible for generating the posterior samples and calling \code{tally()} when it is appropriate to save the model's state. In addition, \code{_loop} should monitor the sampler's \code{status} attribute at every iteration and respond appropriately. The possible values of \code{status} are:
   \begin{description}
      \item[\code{'ready'}:] Ready to sample.
      \item[\code{'running'}:] Sampling should continue as normal.
      \item[\code{'halt'}:] Sampling should halt as soon as possible. \code{_loop} should call the \code{halt()} method and return control. \code{_loop} can set the status to \code{'halt'} itself if appropriate (eg the database is full or a \code{KeyboardInterrupt} has been caught).
      \item[\code{'paused'}:] Sampling should pause as soon as possible. \code{_loop} should return, but should be able to pick up where it left off next time it's called.
   \end{description}
\end{description}

Samplers may alternatively want to override the default \code{sample()} method. In that case, they should call the \code{tally()} method whenever it is appropriate to save the current model state. Like custom \code{_loop()} methods, custom \code{sample()} methods should handle \code{KeyboardInterrupts} and call the \code{halt()} method when sampling terminates to finalize the traces.

\hypertarget{dont-update-indepth}{}
\section{Don't update stochastic variables' values in-place}
\label{dont-update-indepth}
\pdfbookmark[0]{Don't update stochastic variables' values in-place}{dont-update-indepth}

If you're going to implement a new step method, fitting algorithm or unusual (non-numeric-valued) \code{Stochastic} subclass, you should understand the issues related to in-place updates of \code{Stochastic} objects' values. Fitting methods should never update variables' values in-place for two reasons:
\begin{itemize}
   \item In algorithms that involve accepting and rejecting proposals, the `pre-proposal' value needs to be preserved uncorrupted. It would be possible to make a copy of the pre-proposal value and then allow in-place updates, but in PyMC we have chosen to store the pre-proposal value as \code{Stochastic.last_value} and require proposed values to be new objects. In-place updates would corrupt \code{Stochastic.last_value}, and this would cause problems.
   \item \code{LazyFunction}'s caching scheme checks variables' current values against its internal cache by reference. That means if you update a variable's value in-place, it or its child may miss the update and incorrectly skip recomputing its value or log-probability.
\end{itemize}

However, a \code{Stochastic} object's value can make in-place updates to itself if the updates don't change its identity. For example, the \code{Stochastic} subclass \code{gp.GP} is valued as a \code{gp.Realization} object. GP realizations represent random functions, which are infinite-dimensional stochastic processes, as literally as possible. The strategy they employ is to `self-discover' on demand: when they are evaluated, they generate the required value conditional on previous evaluations and then make an internal note of it. This is an in-place update, but it is done to provide the same behavior as a single random function whose value everywhere has been determined since it was created.
