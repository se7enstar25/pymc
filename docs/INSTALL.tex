

PyMC is known to run on Mac OS X, Linux and Windows, but in theory should be able to work on just about any platform for which Python is available (and there are many). Before installing PyMC, you will need to install some dependencies.


%___________________________________________________________________________

\hypertarget{dependencies}{}
\pdfbookmark[0]{Dependencies}{dependencies}
\section*{Dependencies}
\label{dependencies}

PyMC requires some prerequisite packages to be present on the system before the PyMC package itself is installed. Fortunately, there are currently only a few dependencies, and all are freely available online.
\begin{itemize}
\item {} 
\href{http://www.python.org/.}{Python} version 2.4 or later.

\item {} 
\href{http://www.scipy.org/NumPy}{NumPy} version 1.0 or later: The fundamental scientific programming package, it provides a multidimensional array type and many useful functions for numerical analysis.

\item {} 
\href{http://matplotlib.sourceforge.net/}{Matplotlib} (optional): A robust plotting package.

\item {} 
\href{http://www.scipy.org/}{SciPy} (optional): A scientific package with a lot of dtrmity.

\item {} 
\href{http://ipython.scipy.org/}{IPython} (optional): An advanced python shell with many usability enhancements and features such as parallel computing facilities.

\end{itemize}

There are prebuilt distributions that include all the needed dependencies. We recommend the \href{http://www.activestate.com/Products/ActivePython/}{ActiveState} distributions, as they are quality-assured and include several useful modules.

If instead of installing the prebuilt binaries you prefer (or have) to build PyMC yourself, make sure you have a Fortran and a C compiler. There are free compilers (g77, gcc) available on all platforms. Other compilers have not been tested with PyMC but may work nonetheless.


%___________________________________________________________________________

\hypertarget{platform-specific-instructions}{}
\pdfbookmark[0]{Platform-specific instructions}{platform-specific-instructions}
\section*{Platform-specific instructions}
\label{platform-specific-instructions}


%___________________________________________________________________________

\hypertarget{windows}{}
\pdfbookmark[1]{Windows}{windows}
\subsection*{Windows}
\label{windows}
\newcounter{listcnt0}
\begin{list}{\arabic{listcnt0}.}
{
\usecounter{listcnt0}
\setlength{\rightmargin}{\leftmargin}
}
\item {} 
Download the prebuilt binary installer from \href{http://code.google.com/p/pymc/downloads/list}{http://code.google.com/p/pymc/downloads/list}

\item {} 
Simply double-click the executable installation package, and follow the on-screen instructions.

\end{list}

To build from source, get the source code tarball or checkout the subversion repository, move into the resulting source directory in the command terminal and type:
\begin{quote}{\ttfamily \raggedright \noindent
python~setup.py~build~-{}-compiler=mingw32~install
}\end{quote}

This assumes you are using the GCC compiler (recommended). Otherwise, change the -{}-compiler argument accordingly.


%___________________________________________________________________________

\hypertarget{mac-os-x}{}
\pdfbookmark[1]{Mac OS X}{mac-os-x}
\subsection*{Mac OS X}
\label{mac-os-x}
\setcounter{listcnt0}{0}
\begin{list}{\arabic{listcnt0}.}
{
\usecounter{listcnt0}
\setlength{\rightmargin}{\leftmargin}
}
\item {} 
Download the prebuilt binary installer from \href{http://code.google.com/p/pymc/downloads/list}{http://code.google.com/p/pymc/downloads/list}

\item {} 
Double-click the installer package. Your archive utility will usually expand the installer from its archive. From here, the prerequisite packages can be installed (by double-clicking the installers and following the on-screen instructions), followed by the PyMC package itself.

\end{list}

To build and install PyMC from source:
\setcounter{listcnt0}{0}
\begin{list}{\arabic{listcnt0}.}
{
\usecounter{listcnt0}
\setlength{\rightmargin}{\leftmargin}
}
\item {} 
Download the source tarball or checkout the subversion repository.

\item {} 
Untar the source archive, then move into the resulting source directory in the command terminal and type:
\begin{quote}{\ttfamily \raggedright \noindent
python~setup.py~build~\\
sudo~python~setup.py~install
}\end{quote}

\end{list}

The \titlereference{sudo} command is required to install PyMC into the Python site-packages directory, which should have restricted privileges. You will be prompted for a password, and provided you have superuser privileges, the installation will proceed.


%___________________________________________________________________________

\hypertarget{linux}{}
\pdfbookmark[1]{Linux}{linux}
\subsection*{Linux}
\label{linux}

Unfortunately, binary installers are not currently available for Linux systems, but it is straightforward to build the package yourself.
\setcounter{listcnt0}{0}
\begin{list}{\arabic{listcnt0}.}
{
\usecounter{listcnt0}
\setlength{\rightmargin}{\leftmargin}
}
\item {} 
Download the source tarball or checkout the subversion repository.

\item {} 
Untar the package archive, then move to the resulting archive directory and type:
\begin{quote}{\ttfamily \raggedright \noindent
python~setup.py~build~\\
sudo~python~setup.py~install
}\end{quote}

\end{list}


%___________________________________________________________________________

\hypertarget{subversion-repository}{}
\pdfbookmark[0]{Subversion repository}{subversion-repository}
\section*{Subversion repository}
\label{subversion-repository}

You can check out the bleeding edge version of the code from the subversion repository:
\begin{quote}{\ttfamily \raggedright \noindent
svn~checkout~http://pymc.googlecode.com/svn/branches/trial~pymc
}\end{quote}

Previous versions are available in the /tags/ directory.


%___________________________________________________________________________

\hypertarget{running-the-test-suite}{}
\pdfbookmark[0]{Running the test suite}{running-the-test-suite}
\section*{Running the test suite}
\label{running-the-test-suite}

To make sure everything is working correctly, open a python shell and type:
\begin{quote}{\ttfamily \raggedright \noindent
import~PyMC~\\
PyMC.test()
}\end{quote}

You should see a lot of tests being run, and messages appear if errors are raised or if some tests fail. In case this happens (it shouldn't), please report the problems on the issue tracker, specifying the version you are using and the environment. Some of the tests require SciPy, if it is not installed on your system, you should not worry too much about failing tests.


%___________________________________________________________________________

\hypertarget{bugs-and-feature-requests}{}
\pdfbookmark[0]{Bugs and feature requests}{bugs-and-feature-requests}
\section*{Bugs and feature requests}
\label{bugs-and-feature-requests}

Report problems with the installation, bugs in the code or feature request at the issue tracker at \href{http://code.google.com/p/pymc/issues/list}{http://code.google.com/p/pymc/issues/list} .
