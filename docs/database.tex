

In the examples seen so far, traces are simply held in memory and discarded once
the Python session ends. PyMC provides the means to store these traces on disk,
load them back and add additional samples. Internally, this is implemented in
what we call \emph{database backends}. Each one of these backends is simply made
of two classes: \texttt{Database} and \texttt{Trace} which all present a similar
interface to users.
At the moment, PyMC counts seven such backends: \texttt{ram}, \texttt{no{\_}trace},
\texttt{pickle}, \texttt{txt}, \texttt{sqlite}, \texttt{mysql} and \texttt{hdf5}.
In the following, we present the common interface to those backends and a
description of each individual backend.


%___________________________________________________________________________

\hypertarget{accessing-sampled-data}{}
\section{Accessing Sampled Data}
\label{accessing-sampled-data}

To recommended way to access data from an MCMC run, irrespective of the
database backend, is to use the \texttt{trace(name, chain=-1)} method:
\begin{quote}{\ttfamily \raggedright \noindent
>{}>{}>~M~=~MCMC(DisasterModel)~\\
>{}>{}>~M.sample(10)~\\
>{}>{}>~M.trace('e'){[}:{]}~\\
array({[}~2.28320992,~~2.28320992,~~2.28320992,~~2.28320992,~~2.28320992,~\\
~~~~~~~~2.36982455,~~2.36982455,~~3.1669422~,~~3.1669422~,~~3.14499489{]})
}\end{quote}

\texttt{M.trace('e')} returns the \texttt{Trace} instance associated with the tallyable
object \titlereference{e}:
\begin{quote}{\ttfamily \raggedright \noindent
>{}>{}>~M.trace('e')~\\
<pymc.database.ram.Trace~object~at~0x7fa4877a8b50>
}\end{quote}

This \texttt{Trace} object from the \texttt{ram} backend has a \texttt{{\_}{\_}getitem{\_}{\_}} method
that is used to access the trace, just as with any other NumPy array.
By default, \texttt{trace} returns the samples from
the last chain (chain=-1), which in this case is equivalent to \texttt{chain=0}. To
return the samples from all the chains, use \texttt{chain=None}:
\begin{quote}{\ttfamily \raggedright \noindent
>{}>{}>~M.sample(5)~\\
>{}>{}>~M.trace('e',~chain=None){[}:{]}~\\
array({[}~2.28320992,~~2.28320992,~~2.28320992,~~2.28320992,~~2.28320992,~\\
~~~~~~~~2.36982455,~~2.36982455,~~3.1669422~,~~3.1669422~,~~3.14499489,~\\
~~~~~~~~3.14499489,~~3.14499489,~~3.14499489,~~2.94672454,~~3.10767686{]})
}\end{quote}


%___________________________________________________________________________

\hypertarget{saving-data-to-disk}{}
\section{Saving Data to Disk}
\label{saving-data-to-disk}

By default, the database backend selected by the \texttt{MCMC} sampler is the \texttt{ram}
backend, which simply holds the data in RAM memory. Now, we will create a
sampler that, instead, will write data to a pickle file:
\begin{quote}{\ttfamily \raggedright \noindent
>{}>{}>~M~=~MCMC(DisasterModel,~db='pickle',~dbname='Disaster.pickle')~\\
>{}>{}>~M.db~\\
<pymc.database.pickle.Database~object~at~0x7fa486623d90>~\\
~\\
>{}>{}>~M.sample(10)~\\
>{}>{}>~M.db.commit()
}\end{quote}

Note that in this particular case, no data is written to disk before the call
to \texttt{db.commit}. The \texttt{commit} call creates a file named \titlereference{Disaster.pickle}
that contains the trace of each tallyable object as well as the final state of
the sampler. This means that a user that forgets to call the \texttt{commit}
method runs the risk of losing his data. Some backends write the data to disk
continuously, so that not calling \texttt{commit} is less of an issue.

In general, however, it is recommended to always call the \texttt{db.close} method
before closing the session. The \texttt{close} method first calls \texttt{commit}, and
goes further in making sure that the database is in a safe state. Once \texttt{close}
has been called, further call to \texttt{sample} will likely fail, at least
for some backends.
\begin{center}\begin{sffamily}
\fbox{\parbox{\admonitionwidth}{
\textbf{\large Warning}
\vspace{2mm}

Always call the \texttt{close} method before closing the session to avoid running
the risk of losing your data.
}}
\end{sffamily}
\end{center}


%___________________________________________________________________________

\hypertarget{loading-back-a-database}{}
\section{Loading Back a Database}
\label{loading-back-a-database}

To load a file created in a previous session, use the \texttt{load} function
from the backend that created the database:
\begin{quote}{\ttfamily \raggedright \noindent
>{}>{}>~db~=~pymc.database.pickle.load('Disaster.pickle')~\\
>{}>{}>~len(db.trace('e'){[}:{]})~\\
10
}\end{quote}

The \texttt{db} object also has a \texttt{trace} method identical to that of \texttt{Sampler}.
You can hence inspect the results of a model, even when you don't have the model
around.

To add samples to this file, we need to create an MCMC instance. This time,
instead of setting \texttt{db='pickle'}, we will pass the existing \texttt{Database}
instance:
\begin{quote}{\ttfamily \raggedright \noindent
>{}>{}>~M~=~MCMC(DisasterModel,~db=db)~\\
>{}>{}>~M.sample(5)~\\
>{}>{}>~len(M.trace('e',~chain=None){[}:{]})~\\
15~\\
>{}>{}>~M.db.close()
}\end{quote}


%___________________________________________________________________________

\hypertarget{backends-description}{}
\section{Backends Description}
\label{backends-description}


%___________________________________________________________________________

\hypertarget{ram}{}
\subsection{ram}
\label{ram}

Used by default, this backend simply holds a copy in memory, with no output
written to disk. This is useful for short runs or testing. For long runs
generating large amount of data, using this backend may fill the available
memory, forcing the OS to store data in the cache, slowing down all running
applications on your computer.


%___________________________________________________________________________

\hypertarget{no-trace}{}
\subsection{no{\_}trace}
\label{no-trace}

This backend simply does not store the trace. This may be useful for testing
purposes.


%___________________________________________________________________________

\hypertarget{txt}{}
\subsection{txt}
\label{txt}

With the \texttt{txt} backend, the data is written to disk in ASCII files.
More precisely, the \texttt{dbname} argument is used to create a top directory
into which chain directories, called \texttt{Chain{\_}<{\#}>}, are going to be created each
time \texttt{sample} is called:
\begin{quote}{\ttfamily \raggedright \noindent
dbname/~\\
~~Chain{\_}0/~\\
~~~~<object0~name>.txt~\\
~~~~<object1~name>.txt~\\
~~~~...~\\
~~Chain{\_}1/~\\
~~~~<object0~name>.txt~\\
~~~~<object1~name>.txt~\\
~~~~...~\\
~~...
}\end{quote}

In each one of these chain directories, files named \texttt{<variable name>.txt}
are created, storing the values of the variable as rows of text:
\begin{quote}{\ttfamily \raggedright \noindent
{\#}~Variable:~e~\\
{\#}~Sample~shape:~(5,)~\\
{\#}~Date:~2008-11-18~17:19:13.554188~\\
3.033672373807017486e+00~\\
3.033672373807017486e+00~\\
...
}\end{quote}

Although this backend makes it easy to load the data using another application,
for large datasets files tend to be embarassingly large and slow to load
into memory.


%___________________________________________________________________________

\hypertarget{pickle}{}
\subsection{pickle}
\label{pickle}

The \texttt{pickle} database relies on the \texttt{cPickle} module to save the
traces. Use of this backend is appropriate for small scale,
short-lived projects. For longer term or larger projects, the \texttt{pickle}
backend should be avoided since generated files might be unreadable
across different Python versions. The \titlereference{pickled} file is a simple dump of a
dictionary containing the NumPy arrays storing the traces, as well as
the state of the \texttt{Sampler}'s step methods.


%___________________________________________________________________________

\hypertarget{sqlite}{}
\subsection{sqlite}
\label{sqlite}

The \texttt{sqlite} backend is based on the python module \href{http://www.sqlite.org}{sqlite3} (
a Python 2.5 built-in ) . It opens an SQL database named \texttt{dbname},
and creates one table per tallyable objects. The rows of this table
store a key, the chain index and the values of the objects as:
\begin{quote}{\ttfamily \raggedright \noindent
key~(INT),~trace~(INT),~~v1~(FLOAT),~v2~(FLOAT),~v3~(FLOAT)~...
}\end{quote}

The key is autoincremented each time a new row is added to the table.
\begin{center}\begin{sffamily}
\fbox{\parbox{\admonitionwidth}{
\textbf{\large Warning}
\vspace{2mm}

Note that the state of the sampler is not saved by
the \texttt{sqlite} backend.
}}
\end{sffamily}
\end{center}


%___________________________________________________________________________

\hypertarget{mysql}{}
\subsection{mysql}
\label{mysql}

The \texttt{mysql} backend depends on the \href{http://www.mysql.com/downloads/}{MySQL} library and its python wrapper
\href{http://sourceforge.net/projects/mysql-python}{MySQLdb}. Like the \texttt{sqlite} backend, it creates an SQL database containing
one table per tallyable object. The main difference of \texttt{mysql} compared to
\texttt{sqlite} is that it can connect to a remote database, provided the url and
port of the host server is given, along with a valid user name and password.
These parameters are passed when the \texttt{Sampler} is instantiated:
\begin{itemize}
\item {} 
\texttt{dbname} (\titlereference{string}) The name of the database file.

\item {} 
\texttt{dbuser} (\titlereference{string}) The database user name.

\item {} 
\texttt{dbpass} (\titlereference{string}) The user password for this database.

\item {} 
\texttt{dbhost} (\titlereference{string}) The location of the database host.

\item {} 
\texttt{dbport} (\titlereference{int})    The port number to use to reach the database host.

\item {} 
\texttt{dbmode} {\{}\texttt{a}, \texttt{w}{\}} File mode.  Use \texttt{a} to append values, and \texttt{w}
to overwrite an existing database.

\end{itemize}
\begin{center}\begin{sffamily}
\fbox{\parbox{\admonitionwidth}{
\textbf{\large Warning}
\vspace{2mm}

Note that the state of the sampler is not saved by
the \texttt{mysql} backend.
}}
\end{sffamily}
\end{center}


%___________________________________________________________________________

\hypertarget{hdf5}{}
\subsection{hdf5}
\label{hdf5}

The \texttt{hdf5} backend uses \href{http://www.pytables.org/moin}{pyTables} to save data in binary HDF5 format.
The \texttt{hdf5} database is fast and can store huge traces, far larger than the
available RAM. This data can be compressed and decompressed on the fly to
reduce the memory footprint.
Another feature of this backends is that it can store arbitrary objects.
Whereas the other backends are limited to numerical values, \texttt{hdf5} can
tally any object that can be pickled, opening the door for powerful and
exotic applications (see \texttt{pymc.gp}).

The internal structure of an HDF5 file storing both numerical values and
arbitrary objects is as follows:
\begin{quote}{\ttfamily \raggedright \noindent
/~(root)~\\
~~/chain0/~(Group)~'Chain~{\#}0'~\\
~~~~/chain0/PyMCSamples~(Table(N,))~'PyMC~Samples'~\\
~~~~/chain0/group0~(Group)~'Group~storing~objects.'~\\
~~~~~~/chain0/group0/<object0~name>~(VLArray(N,))~'<object0~name>~samples.'~\\
~~~~~~/chain0/group0/<object1~name>~(VLArray(N,))~'<object1~name>~samples.'~\\
~~~~~~...~\\
~~/chain1/~(Group)~'Chain~{\#}1'~\\
~~~~...
}\end{quote}

All standard numerical values are stored in a \texttt{Table}, while \texttt{objects}
are stored in individual \texttt{VLArrays}.

The \texttt{hdf5} Database takes the following parameters:
\begin{itemize}
\item {} 
\texttt{dbname} (\titlereference{string}) Name of the hdf5 file.

\item {} 
\texttt{dbmode} {\{}\texttt{a}, \texttt{w}, \texttt{r}{\}} File mode: \texttt{a}: append, \texttt{w}: overwrite,
\texttt{r}: read-only.

\item {} 
\texttt{dbcomplevel} : (\titlereference{int} (0-9)) Compression level, 0: no compression.

\item {} 
\texttt{dbcomplib} (\titlereference{string}) Compression library (\texttt{zlib}, \texttt{bzip2}, \texttt{lzo})

\end{itemize}

According the the \href{http://www.pytables.org/moin}{pyTables} manual, \titlereference{zlib} has a fast decompression,
relatively slow compression, and a good compression ratio.
\titlereference{LZO} has a fast compression, but a low compression ratio.
\titlereference{bzip2} has an excellent compression ratio but requires more CPU. Note that
some of these compression algorithms require additional software to work (see
the \href{http://www.pytables.org/moin}{pyTables} manual).


%___________________________________________________________________________

\hypertarget{writing-a-new-backend}{}
\section{Writing a New Backend}
\label{writing-a-new-backend}

It is relatively easy to write a new backend for \texttt{PyMC}. The first step is to
look at the \texttt{database.base} module, which defines barebone \texttt{Database}
and \texttt{Trace} classes. This module contains documentation on the methods that
should be defined to get a working backend.

Testing your new backend should be trivial, since the \texttt{test{\_}database}
module contains a generic test class that can easily be subclassed to check
that the basic features required of all backends are implemented and working
properly.

