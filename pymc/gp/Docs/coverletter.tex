\documentclass{letter}

% \usepackage{garamond}
\usepackage[colorlinks]{hyperref}
% \usepackage[T1]{fontenc}

\signature{Anand Patil}
\address{Malaria Atlas Project\\
    Department of Zoology\\
    University of Oxford\\
    Oxford, OX1 3PS, UK} 

\begin{document}
    % \garamond

\begin{letter}{Jan de Leeuw and Achim Zeileis \\ Editors \\ Journal of Statistical Software}
\opening{Dear Mr de Leeuw and Mr. Zeileis,}

I am pleased to submit a paper presenting a Gaussian process package for the Bayesian analysis package PyMC. PyMC's user guide is due to be published shortly in JSS as `PyMC 2: Bayesian stochastic modeling in Python' by Patil, Huard and Fonnesbeck.

I wrote the Gaussian process package in order to showcase the advantages of PyMC, and Python more broadly, for Bayesian analysis. It aims to provide a uniquely convenient and intuitive interface by using an object model that closely resembles the underlying mathematical concepts, while still allowing nearly unlimited flexibility in model construction and achieving good performance. I have used the package to conduct about a dozen geostatistical analyses in my work at the Malaria Atlas Project.

The Gaussian process package is distributed with PyMC, but I have maintained it semi-independently and it is not covered by PyMC's documentation. I am confident that you will agree it represents a substantial contribution over and above PyMC itself, and deserves publication on its own.

This paper is a condensed version of the package documentation, which is available from \href{http://pymc.googlecode.com/files/GPUserGuide.pdf}{PyMC's homepage on Google Code}.

\closing{Sincerely,}

\end{letter}

\end{document}

% \documentclass{letter} 
% \name{Anand Patil} 
% \signature{Anand Patil}
% 
% \begin{document}
% \begin{letter}
% \opening{Dear Mr. de Leeuw,} 
% blah
% \closing{Sincerely,} 
% \end{letter}    
% \end{document}
